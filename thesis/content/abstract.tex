% !TEX root = ../my-thesis.tex
%
\pdfbookmark[0]{Abstract}{Abstract}
\addchap*{Kurzfassung}
\label{sec:abstract}
Die freie Wissensdatenbank Wikidata bietet Exporte der Daten als RDF an.
Die Größe dieser Exporte stellt jedoch ein Hindernis für die Verarbeitung dar.
Auf Basis von Wikidata Toolkit wird in dieser Arbeit eine Anwendung entwickelt, die Dumps für Teile der Daten generiert.
Die Anwendung erlaubt es den Nutzern, eigene Kriterien zum Filtern der Exporte anzugeben.
Zur Überprüfung der Konsistenz dieser Dumps mit den vollständigen Exporten wird ein Vergleich durchgeführt.
Dabei wird das von Wikidata Toolkit erzeugte RDF mit den Wikidata-Exporten verglichen.
Mit dieser Methode werden mehrere Fehler in Wikidata Toolkit aufgedeckt.
Durch Beheben dieser Fehler leistet die Arbeit einen Beitrag zur Verbesserung des RDF-Exports von Wikidata Toolkit.
Mit der entwickelten Anwendung wird die Verwendung von Daten aus Wikidata vereinfacht.

\vspace*{20mm}

{\usekomafont{chapter}Abstract}
\label{sec:abstract-diff}

Wikidata, a free knowledge base, provides data exports in RDF.
However, usage of these exports is a challenge due to their large size.
In this work we present an application based on Wikidata-Toolkit that solves this problem by providing dumps of parts of the data.
User-defined criteria allow fine-grained filtering of the exports.
To verify the consistency of dumps generated this way, we perform a comparision.
In this comparision, RDF as generated by Wikidata-Toolkit is compared to the full RDF exports provided by Wikidata.
Multiple issues in Wikidata-Toolkit are uncovered.
In addition to developing a useful tool for consumers of Wikidata exports, we contribute several fixes to Wikidata-Toolkit.