\documentclass[aspectratio=1610,xcolor=svgnames]{beamer}


\usepackage{appendixnumberbeamer}
\usepackage{tikz}
\usepackage{varwidth}
\usepackage[absolute,overlay]{textpos}
\usepackage{pstricks}
\usepackage[ngerman]{babel}

\usetheme{metropolis}

\usetikzlibrary{patterns}
\usetikzlibrary{shapes}
\usetikzlibrary{shapes.symbols}
\usetikzlibrary{fit}
\usetikzlibrary{backgrounds}
\usetikzlibrary{calc}

\newcommand{\tikzdbg}{%
  \draw[step=1,color=lightgray] (-7,0) grid (7,7);%
  \fill[red] (0,0) circle (0.2);%
}


\typeout{CONTENT START NOW}

\title{Generierung angepasster RDF-Dumps von Wikidata}
\subtitle{Bachelorverteidigung}
\date{\today}
\author{Benno Fünfstück}
\institute{Betreuer: Prof. Dr. Markus Krötzsch \\ Wissensbasierte Systeme \\ TU Dresden}

\begin{document}

\begin{frame}
\maketitle
\end{frame}

\begin{frame}\frametitle{Wikidata}
\end{frame}

\begin{frame}\frametitle{Größe von Wikidata}
\end{frame}

\section{Idee: Tool zum Filtern der Daten}

\begin{frame}\frametitle{Beispiele}
\end{frame}

\begin{frame}\frametitle{Anforderungen}
\end{frame}

\section{Umsetzung}

\begin{frame}\frametitle{Architekturvarianten}
\end{frame}

\begin{frame}\frametitle{UI-Konzept}
\end{frame}

\section{Evaluation}

\begin{frame}\frametitle{Korrektheit}
\end{frame}

\begin{frame}\frametitle{Verwandte Arbeiten}
\end{frame}

\begin{frame}\frametitle{Fazit}
\end{frame}

\appendix

\begin{frame}\frametitle{Referenzen}
\end{frame}


\end{document}
